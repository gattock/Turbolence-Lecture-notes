\documentclass[a4paper,11pt]{article}
\usepackage{amsmath}
\usepackage{amsfonts}
\usepackage{physics}
\usepackage {graphicx}
	\usepackage{subfigure}
\usepackage{geometry}
\usepackage{cancel}
\usepackage{color,xcolor,soul}
\usepackage{enumitem} %itemize settings
\setitemize{noitemsep,topsep=0pt,parsep=0pt,partopsep=0pt}
\geometry{left=2.5cm,right=2.5cm,top=2.5cm,bottom=2.5cm}

\newcommand{\mathcolorbox}[2]{\colorbox{#1}{$\displaystyle #2$}}
\newcommand{\pvir}{\; ; \;} % pvir=punto e virgola in italian is this-> ;
\newcommand{\RA}{\Rightarrow}
\newcommand{\dive}{\nabla \cdot}
\newcommand{\rot}{\nabla \times}
\newcommand{\cic}[1]{\mathbf{#1}}
\newcommand{\avg}[1]{\langle #1 \rangle}

	\begin{document}
\title{%
turbulence and Mixing course by\\
Professor Jaan Kalda, TalTech\\
lecture notes by}
\author{student Enrico Di Lavore}
\date{26 November 2021}
\maketitle

%%%%%%%%%%%%%%%%%%%%%%%%%%%%%%%%%%%%%%%%%
	\section{Lect.1: Intro on main concepts and classification of turbulence}
	
\subsection {Trajectories:}
Also called Pathlines, are typically a lagrangian idea. Are the trajectories of the individual fluid particles. 
Those can be computed integrating the equation
\[ d\cic{x}=\cic{v}(\cic{x},t) dt   \]
given the initial positions.
The direction of the trajectory is locally determined by the streamlines of the fluid at each moment in time.

\subsection{Streamlines:}
Are the family of curves that are instantaneously \textbf{tangent} to $\cic{V}$ vector field. These show the direction in which a massless (without inertia) fluid element would travel at any point in time.
Streamlines are defined by:
\[ \dv{\cic{x_s}}{s}  \times \cic{v}(\cic{x_s})=0  \RA \dv{\cic{x_s}}{s}\parallel  \cic{v}(\cic{x_s})   \] \\
where s indicate the particular streamline (curvilinear abscissa) . Intuitively, the cross product equal to zero impose the two vectors to be \textbf{parallel}. From here the tangential direction.\\

\begin{figure}[ht]
\centering
\includegraphics [scale=0.3] {pic1.1streamlines.jpg}
%\caption{Streamlines}
\end{figure}

\footnote{The reader is suggested to check also Streamtube; Streamlines and timelines which provide a snapshot of some flowfield characteristics, whereas streaklines and pathlines depend on the full time-history of the flow}
Note that if the flow is \textbf{stationary} then all those lines coincide. 

\hl{check}

\subsection {Turbulence:} 
In fluid dynamics, a turbulent flow is fluid motion characterized by \textbf{chaotic changes} in pressure and flow velocity. It cause mixing.\\
The onset of turbulence can be predicted by the dimensionless \textbf{Reynolds number} $Re$, the ratio of kinetic energy to viscous damping in a fluid flow. $Re=\frac{v \cdot L}{\nu}$

\begin{figure}
    \centering
    \subfigure[]{\includegraphics[width=0.6\textwidth]{pic1.2turbulence.png}} 
    \subfigure[]{\includegraphics[width=0.24\textwidth]{pic1.1flow.png}} 
    \caption{(a) Turbulent flow (b) Laminar and turbulent flow in pipes}
    %\label{fig:foobar}
\end{figure}


\subsection{Laminar flow:}
 It occurs when a fluid flows in parallel layers, with no disruption between those layers.
 The shear forces between stream lines cause a typical speed function. 
 It is characterised by small values of $Re$ number.\\
Can be seen in the picture that the speed is constant everywhere and decreasing with a certain function close to the plumb surface. \\
 This is known as \textbf{boundary layer}.
 %attach boundary layer graph pic

\subsection{Mixing:}
It is a \textbf{random transport} of \emph{something} (passive or active, scalar or vector field) \\ %attach turbulence pics
Vectorial field case \textbf{diffusion equation}:
\[ \dv{\cic{n}}{t} =\pdv{\cic{n}}{t}+(\cic{v} \nabla)\cic{n} = D \cdot \Delta \cic{n} \]
-The first term $\dv{n}{t}$ is the Lagrangian (material) derivative \\
-In the second term $\pdv{n}{t}$ is the Eulerian derivative. \\

\hl{expand and explain:} In case of voriticity, the random transport is affected by a Stretching-Folding re-iteration, that provide a stratified structure of density values in the domain.

\begin{figure}[ht]
\centering
\includegraphics [scale=1] {pic1.2mix.jpg}
\end{figure}

\begin{figure}[ht]
\centering
\includegraphics [scale=2.5] {pic1.2fold.jpg}
\caption{Stretch and fold iteration}
\end{figure}

\subsubsection{Why should diffusion be considered as a random transport?}
\hl{Insert: heat stochastic theory, proabilistic approach on evaporation,photo electric effect...}


\subsection{Advection:}
The $advection$ term is often used as synonim for $convection$. More technically, convection applies to the movement of a fluid by buoyancy force (often due to density gradients created by thermal gradients), whereas advection is the movement of some material by the generic velocity of the fluid.
We consider here advection of a scalar quantity $\psi$ in $n$dimensions :
\[   \pdv{\psi}{t}+\dive (\psi \cic{v})=0   \]
What this equation means? We analyse the 1 dimensional case:
The flux density $\Phi$  of the quantity $\psi$ is $\Phi=c\psi$ where c is the drift speed of a substance e.g. in a tube with flowing liquid. $\psi$ represent the concentration of the substance. Substituting this equation into local conservation law, we get the \textbf{transport equation} for the concentration $\psi$:
$\pdv{\psi}{t}+c\psi_x=0$ \\
 From the differential analysis you know that $\psi(x,t)=F(x-ct)$, that is a travelling wave. \\
 You can see that the profile of function $\psi(x)$ is rigidly shifting.

\begin{figure}[ht]
\centering
\includegraphics [scale=0.5] {pic1.4transport.png}
\caption{Example: transport equation solution}
\end{figure}

 
 \subsection{Transport \& diffusion equation in n-dim:}
 The reader can have fun (if haven't had in the past) to obtain the transport and diffusion equation in the $n-$ dimensional case: considering the quantity $\psi$ it is
 
 \[  \pdv{\psi (\cic{x},t)}{t}= \cic{c} \nabla \psi (\cic{x},t) - k \Delta \psi (\cic{x},t)  \]
 
\subsection{Passive: vector field or scalar}
 A vector field $\cic{A}$ (or a scalar $A$) is said to be passive if the velocity field does \textbf{not depend} on $\cic{A}$ (or $A$) mixing state evolution.
 $ \cic{v} \neq f(\cic{A}(\cic{X},t) $  ( or $ \cic{v} \neq f(A(\cic{X},t) $ )
	\subsubsection{e.g. Passive scalar:}
If we consider heat diffusion in a mixing process, in Hp. of \emph{incompressible} fluid we can write that  $ \rho(T)=const. $ \\
In this case T° is an attribute of the \emph{fluid particle} that does not influence the motion of the fluid itself. \\
Other examples of scalars that are good to be modeled as passive: \\
$\cdot$ \emph{Dye concentration} $\cic{n}$ in a fluid (which values as $\rho , \nu ...$ must be similar enough to the fluid ones in order to not influence the time evolution) is considered passive if its concentration is low enough. \\
$\cdot$ \hl{ Others? check}
	\subsubsection{e.g. Passive vector field:} \textbf{prof.check}
If we consider electro-magnetic field in a medium (as water) while studing ships propellers phenomena as cavitation or flutter, then is a good Hp. to consider E-M field as a \textbf{passive vector field}. 
Indeed it is not influencing noticeably the solution $\cic{v}(\cic{X},t)$ of this particular problem. 
\hl{ not the best example: could be misunderstood with the fact that EM is useless quantity in this problem. By the way, if it was active, then it should be considered.}
 
 \subsection{Active: vector field or scalar}
 A vector field $\cic{A}$ or scalar $A$ is said to be active if the velocity field does \textbf{depend} on $A$ mixing state evolution.
  $ \cic{v} = f(\cic{A}(\cic{X},t) $  ( or $ \cic{v} = f(A(\cic{X},t) $ )
	\subsubsection{e.g. Active scalar:}
We consider again $T$ in a fluid. This time the latest is said \emph{compressible} if 
$ \rho(T°) \neq const. $\\
As in convection phenomena, $\nabla\rho$ generates buoyancy forces among the fluid particles. In this case, $T$ is influencing the motion through density.
	\subsubsection{e.g. Active vector field:} \textbf{prof.check}
$\cdot$ If we are considering a star, here we must consider the E.M. field propagating through the plasma as an active vector field: 
indeed it generates reciprocal forces between two infinitesimal volumes of plasma $dV_1$ and $dV_2$. \\
This forces generate accelerations which, integrated over time, influence $\cic{v}$ field.
This will be seen in a later chapter regarding Magneto-Hydro-Dynamics (MHD).
By the way, if the field is small in the (so called) kinematic magnetic dynamo, then it can be considered passive. 
Magnetic dynamo equation: \[ [\pdv{}{t}+(\cic{v}\nabla)] \cic{B}=D \Delta \cic{B} \]

\hl{Why B field exist in stars, planets, where I comes from? Broken instable equilibria flowing to stability R->0}\\

$\cdot$The angular speed $\omega$ is active vector field by definition: indeed
\[  \omega=\nabla \times \cic{v} \]
turbulence, as the mixing problem of an active field, is the last unsolved problem of classical physics: the vorticity.

\subsection{Mixing of passive scalars:}
\[ \pdv{n}{t}+(\cic{v} \nabla)n = D \cdot \Delta n \]
We can define compressibility P as $P=\frac{\avg{ v_{pot}^2}}{\avg{ v_{pot}^2}+\avg{ v_{sol}^2}}$\\
where \emph{pot} stands for \textbf{potential component} and \emph{sol} for \textbf{solenoidal component}. We see that $P \in [0;1]$, (0 means incompressible).

\subsubsection{Helmoltz decomposition analogy:}
What we have just seen comes from the decomposition of the velocity vector field into its solenoidal component (vector potential) \textbf{A} and potential component $\phi$
\[  \cic{v}=\nabla \times \cic{A}-\nabla\phi \]
This is analogue to see first and second Maxwell equations in the solenoidal-potential framework:
\[  div(\cic{B})=\dive \cic{B}=0 \RA \cic{B}=\nabla\times \cic{A} \]
\[  \nabla \times \cic{E}=0 \RA \cic{E}=-\nabla\phi  \]


\hl{Find references: Weak turbulence? Perturbation theory on lin. solution (e.m. sounds, fluid sounds..)
Strong turbulence: we have coherent structures and non linearity. coherency}

\subsection{Sound propagation}
The set of equations involved into the sound propagation is composed by:

\subsubsection{Continuity equation:}
\[  (\pdv{}{t}+\dive \cic{v})\rho=0 \RA \pdv{\rho}{t}=- \rho ( \dive \cic{v})     \] where $|v|<<C_s (sound speed)$

Its meaning is related with Divergence theorem: \textbf{check, explanation}
we can see that the time derivative of the density is equal in modulus to the velocity vector field divergence times the density. 
If we see the intuitive meaning of the divergence of the velocity vector field (i.e. how much a point behave as a source or sink), by multiplying by the density, we get how much the point behaves as``mass source `` (or sink). 
That is the time derivative of the local density.
\subsubsection{ Newton $2^{nd}$ law (Linear momentum balance law):}
\[   (\pdv{}{t}+\cic{v}\nabla)\cic{v}=- \frac{\nabla p}{\rho} \]
Basically it is $a=\frac{F}{m}=\frac{F}{\rho \cdot V}$ with volume V$\rightarrow 0$; with the acceleration explicited in Eulerian form on the left-hand side
\subsubsection{Adiabatic law: $p \rho^{-\gamma}=const.$}
Hp. the air parcel, subjected to repeated compression-decompression cycles, does not exchange heat with its neighbour parcels; although in reality heat is actually flowing from the high pressure zones (that have just increased their temperature adiabatically starting from common average temperature) to the low pressure zones (with lower temperature for the same reason).
\hl{Feynman diagrams, perturbation theory: linear solution??}

\subsection{Langmuir waves in 2D:}
Let $G_e^-$ be a set of electrons laying on a rigid square grid (frame) of $n \times m$ elements $(n,m) \in \mathbb{N}$. Similarly, let $G_p^+$ be a set of protons laying on another square grid of  $n \times m$ elements. \\
We overlap those two grids, but with certain phase-shifts along two axis: $\lambda_x$ and $\lambda_y$. \\ (See picture). 

\begin{figure}[ht]
\centering
\includegraphics [scale=0.5] {pic1.4langmuirwaves.jpg}
\caption{Langmuir waves charges grids}
\end{figure}

We can intuitively see that the minimum of the electro-static potential happens when the phase-shifts $\lambda_x=\lambda_y=0$. 
This is the global minimum. (The proof is left to the reader). \\
Also, the electrostatic potential will increase while getting away from the main overlapping mode ( when $\lambda_x=\lambda_y=0$ and boundaries of the grid also coincide).
The reader can prove that there will be also others equilibria, corresponding to a shift equal to the grid spacing. \\ The latest are local minima.\\ \hl{attach sketch? }

This system, if left with I.C. $\lambda_x \neq 0 $ or $\lambda_y \neq 0 $ will oscillate ``as a pendulum'' around the minimum. 
Beware to the fact that this is not exactly an harmonic oscillator, since the electrostatic force $f_{q_1 q_2} \propto r_{1,2}^{-2}$, instead the spring force is linear.
The same idea can be applied to not-squared grids (e.g. exagonal grid) and in more than 2 dimensions.
\hl{rigid n body proble, if give a mass, will get kepler trajectories (with some I.C. ellipses) ?}

\subsubsection{$n,m \rightarrow \infty$ case} 
We can also intuitively see that with $n,m \rightarrow \infty$, the function electric potential $U=U(\lambda_x,\lambda_y)$ tend to be a periodic function, and every of the $\infty^2$ minima will be at the same potential $U$.

\subsubsection{Modulation instability: not complete}
\textbf{Attach picture of stable and unstable}
Stable is sol of non-linear Shroedinger equation. See graph.
Important for optical cable communication: solitons=very short pulse
\textbf{solitons=? see travelling waves packets (QM MIT) }

\subsection{Overdamped motion of brownian particle in energy landscape: not complete}
We consider an electron in an electric field. It holds that: \\
$ \cic{F}=-e \nabla \phi $ and $ \cic{v}=-b e \nabla \phi$
We need a parametric evaluation about when a fluid is compressible or not. \\
The parameter that we are looking for is Mach number $Ma$.


%%%%%%%%%%%%%%%%%%%%%%%%%%%%%%%%%%%%%%%%%
\newpage
	\section{Lect.2: Advection and diffusion with stationary flow field}
	
\subsection{Double way to represent advection and diffusion}
Those two ways are equivalent, the convenience depends on the situation.\\
The first is a non-linear ODE, the latter is a linear PDE.
		\subsubsection{as a set of brownian particles}
\[ \dot{\cic{r}}=\dot{\cic{f}}(t)+\cic{v}(\cic{r}) \]
		\subsubsection{as the concentration of a tracer}
\[  \pdv{n}{t}+(\cic{v}\nabla)n=D \Delta n \]

\subsection{Analysis as in high school:}
Starting with just a diffusion along x, the simplest $1-dim$ random walk is with random unit steps done in unit time. $\dot{r}=f(t)$ . \\
By time-discretization we get: $r_{t+1}-r_{t}=s_t=\pm 1  $ (random) \\
The current coordinate $r$ is given by:  $r(t)=\sum_{i=1}^{t}s_i$    	\\
We can compute the average of the random walk in a time interval:
\[ \avg{ r^2(t)}= \avg{\sum_{i=1}^{t}s_i \sum_{j=1}^{t}s_j}=\avg{\sum_{i,j=1}^{t}s_i s_j}= \sum_{i,j=1}^{t}\avg{ s_i s_j}= \sum_{i,j=1}^{t} \delta_{ij}=t   \]
where $\delta_{ij}$ is Kroneker delta.\\
If we declare a length step $\lambda$ and a time step $\tau$, this can be rewritten as: $\avg{ r_t^2}=\frac{\lambda^2 t}{\tau}$ \\
Then the diffusivity $D$ is defined as $D=\frac{\lambda^2}{2 \tau}$, so that $\avg{ r_t^2}=2Dt$\\

\subsection{Description by mean of random white-noise forcing}
We consider the auto-correlation function: still with $\dot{r}=f(t)$, we compute:
\[  r=\int_{0}^{t} f(t')\,dt' \RA r^2=\int_{0}^{t} \int_{0}^{t} f(t')\,f(t'')\,dt'\,dt'' \RA  \]
\[   \RA \avg{ r^2}=\int_{0}^{t}\int_{0}^{t} \avg{ f(t')f(t'')}    \]
But we know that $\avg{ f(t)f(t')}=2D \delta (t-t')$, then we substitute and get:
\[ \avg{ r^2}=\int_{0}^{t}\int_{0}^{t}2\,D\, \delta(t'-t'')\,dt'\,dt''= \int_{0}^{t} 2D\,dt'=2D\,t\]

\newpage%%%%%%%
 \subsection{Derivation of advection-diffusion equation}
 Let's continue by obtaining the PDE from the ODE: \\
 Hp. the I.C. of the function $n(x,t)$, that is $n(x,0)$, is smooth enough so that we can expand it into Taylor series. (Series of order $n$, the function must be $(n+1)$-differentiable).\\
 The idea is to average the final positions of the brownian particle trajectories over the evolution of the random walks. \hl{require clarification+improvement}
\[\dot{r}=v+f(t) \RA \avg{\dot{r}}= \avg{}=v\footnote[1]{Need explanation of this formula:why did that?}  \]

 Given the initial distribution $n(x,0)$ , $\forall\,t\,in\,[0;\varepsilon]$ with $\varepsilon$ sufficiently small, we compute:
 	\footnote[11]{here $\Delta$ is a finite variation of its argument, not the Laplacian!}
 	\footnote[12]{$n_0$ is a constant, its average is itself}
 	\footnote[13]{$\avg{\Delta r^2}=2\,D\cdot t$}
 	\footnote[14]{$n'=\dv{n}{r}$}
 	
\[n(x,t)=\avg{ n_0 \Bigg( r-\int_0^t v+f(t')\,dt \Bigg) }= \avg{ n_0(r)}-\avg{ \dv{n}{r}\cdot \Delta r}+\avg{\dv[2]{n}{r} \frac{\Delta r^2}{2}}=\]
\[ n_0(r)-n'_0\avg{\Delta r}+\frac{n_0}{2}\avg{\Delta r^2} = n_0(r)-n'_0\,v\,t+D\,n''_0\,t \RA \pdv{n_0}{t}t+n'_0\,v\,t=D\,n''_0\,t \RA \pdv{n}{t}+v\pdv{n}{r}=D \pdv[2]{n}{r} \]

And we get the formula for the flux $J$, and we extend its meaning to higher dimension:
\[ \cic{J}=-D \cdot (\nabla n)+n\,\cic{v}\]

Where the first member ($-D \cdot (\nabla n)$) is the dye flux caused by concentration gradient. The second member ($n\,\cic{v}$) is caused by fluid motion.\\
We can apply now the Continuity condition, imposing the dye mass conservation:
\[ \pdv{n}{t}+\dive \cic{J}=0 \]

Now we can substitute $\cic{J}$ into B.C. to get: 
$\dive \cic{J}=\dive \big( -\Delta (\nabla n)+n\cic{v} \big)$  .
Hp. Incompressible flow 
	\footnote[22]{Here $Delta$ is clearly again Laplacian}
\[ \RA \dive \cic{v}=0 \RA \dive (n\cic{v})=(\dive \cic{v})\cdot n +\cic{v}\nabla n=-D\, \Delta n+ \cic{v} \nabla n \]

Here we see that the diffusion equation need at least one among initial and B.C.s to get the solution after infinite time (time-asympthotical solution).\\
To get the time evolution we need the I.C. , and the B.C. if given in the problem.\\ 
\hl{add some theory about diffusion resolution method}\\

$\cdot'$ e.g. we can consider an I.C. like the Dirac delta function $\delta(x,y)$.
This situation represents a finite amount of dye concentrated in a material point, that leads to infinite concentration. This dye will spread over space in time ''by Green Function'', keeping constant its volume integral (i.e. the dye mass is conserved), while its occupied volume expands.



\subsection{Problem solution by mean of Green function:}
Now we need initial or B.C.s. Since the advective-diffusive PDE is linear, its initial value problem can be solved by mean of Green function. \\
In this case,for $v=0$ we would have 
\[ G(t)= \frac{A}{\sqrt{t}} e^{-\frac{x^2}{4D\,t}} \] 
and we would get that $n(x,t)=\int n(x',0)\,G(x-x',t)dx'$ .\\
Where the constant A is found by normalization condition: $\int n(x)\,dx=1$

\subsubsection{Conservation equation:}
In this situation we can impose that the dye mass have to conserve (dye doesn't disappear from universe, hence the its density integral in the whole dominium is constant). We get it by applying Gauss theorem to the diffusion equation: this is done in the following passages.\\
\[\pdv{n}{t}+\cic{v} \nabla n=D \Delta n \RA \int \pdv{n}{t}+\dive \cic{J}=0 \ RA
\pdv{}{t} \int n d^3 \cic{r}+\int \dive \cic{J} d^3 \cic{r}=0\]
\[ \RA \int \dive \cic{J} d^3 \cic{r}= \oint \cic{J} d\cic{A}=0 \RA 
\pdv{}{t}\int_{-\infty}^{\infty} n\,d^3\cic{r}=0  \]

\subsubsection{Theorem, Unknown Name --find it--}
\hl{Refine}: Any flow will always increase the flux from plate 1 to plate 2 (if compared with no flow). 
? Can also happen that if a high concentration gradient is in opposite direction to a very slow flow, the concentration gradient can ''climb the flow'' \hl{as salmons does with rivers?} \\
In case you had some doubts about some passages along the last parts, let's recall some theory. \\

\fbox{\begin{minipage}{40em}
	\subsubsection{Theory remark 1: Green Function}
In functional analysis, the Green function associated to a linear differential operator is the input function of the operator that gives back the \textbf{elementary impulse}, known also as Dirac delta $\delta$. \\
Pratically speaking, it means that if $\cic{L}$ is a linear differential operator, then the Green function $G$ solves the equation $\cic{L}G=\delta$. \\
The physical meaning of this? Have fun ;)

 	\subsubsection{Theory remark 2: Properties of Divergence operator}
 Divergence is a linear operator, hence for every couple of vectorial fields $\cic{f}$ and $\cic{g}$ it holds:\\
 $\dive(a \cic{f}+b \cic{g})=a\,(\dive \cic{f})+b\,(\dive\cic{g}) \pvir \forall\,a,b\,\in \mathbb{R}$ \\
  and also for every function $\varphi(\cic{x})$ it holds that:\\
 $\dive (\varphi\cic{f})=\nabla(\varphi)\cdot\cic{f}+\varphi \, \dive (\cic{f})$\\
 $\dive (\cic{f} \times \cic{g})=\cic{g}\cdot \rot \cic{f} -\cic{f}\cdot\rot \cic{g}$\\ \end{minipage} }


 \subsection{Effective diffusivity: Zeldovich proof (1937)}
 Since this argument is a bit complicated, I report the formulation from the book ''Chaotic Flows'' \footnote[22]{Oleg G. Bakunin, Springer. In the chapter 2.6 is reported the Zeldovich scaling for Effective Diffusivity. On the previous chapter Bakunin defined the minimum value of effective diffusivity on the basis of the fluctuation-dissipation relation.}
The aim is to find an analytic expression of the effective diffusivity. Zeldovich proceeded in the following way.\\
We consider the quasi-steady turbolent flow: the steady scalar density equation is
\[ D_0 \Delta n (\cic{r})-\cic{v}\nabla n (\cic{r})=0 \]

 We apply the simplified perturbation method, that approximate that:
 \[\begin{cases} n=\avg{ n}+n_1=n_0+n_1 \\ \cic{v}=\avg{ \cic{v} }+\nu_1=\nu_1 \end{cases} \]
 because the average of the speed is assumed zero, and the density perturbation is at least two order smaller than the density itself: $n_1<<_0$. Also, $D_0\Delta n_0=0$. We get:
 \[D_0 \pdv[2]{n_1(x)}{x}=\nu_1 \pdv{n_0(x)}{x}. \]
  For sake of simplicity have been presented the 1-dimensional form, so that we can estimate the magnitude:
 \[ n_1 \approx \nu_1 \frac{L_0\,n_0}{D_0} \approx n_0 \, Pe \propto \cic{v}_0 \]
 Here $Pe=\frac{v_0 \, L_0}{D_0}<<1$ is the Peclet number. This Hp. corresponds to weak turbulence assumption.\\
 By deriving this relation, we use the condition of smallness of the term $\nu_1 \nabla n_1$ in comparison with $\nu_1 \nabla n_0$. Then the expression for effective diffusion coefficient is:
 \[ D_eff \approx \frac{1}{n_0^2\,L_0}\int_W D_0 (\nabla\,n_0)^2 \, 
 (1+const \cdot Pe^2)\,dW  \]
 We can obtain the Zeldovich scaling as:
 \[D_eff \propto D_0 \, (1+const \cdot Pe^2)\]
 
 \subsubsection{A bunch of numbers}
 In case of atmospheric turbulence we have $D_0 \approx 0.1 \, [cm^2 \, s^{-1}] \pvir V_0=10 \, [cm \, s^{-1}] \pvir L_0 \approx 10^{-2}\,[cm] \pvir D_eff \approx 10^3 [cm^2 1,s^{-1}]>>D_0$
 This upper estimate of transport $D_eff$ in the steady turbolent flow is given by the scaling:
 \[ D_{eff} \approx v_0^2 \tau \propto v_0^2 \] where $Pe<1$ and the \textbf{correlation time} $\tau$ has a diffusive nature $\tau \approx \tau_D \approx L_0^2 / D_0$.\\
 This result shows the important dependence of the effective diffusivity on the turbulent fluctuation amplitude $v_0$, in the Hp. of small Peclet numbers.
 

 \subsection{Effective diffusivity: class approach}
 We will show that any stationary flow increases the particle density flux (hence the effective diffusivity). The particle density flux is $\cic{J}=\cic{v}\,n-D\nabla n$\\
 We take its dot product with $\nabla n$ and integrate over a volume. We assume as B.C.s that there are two surfaces at which the concentration is kept at constant values $n_1$ and $n_2$, that intuitively are respectively the source and sink of the tracer.\\
 The flow is bounded into a vessel, hence there is no particle flux through the walls of the vessel. We can compute then: \hl{(CHECK COMPUT.)}
 
 \footnote[42]{Remember from properties of the gradient, that by chain rule we get $n\nabla n=\nabla n^2/2$ } 
 
 \[ \begin{cases}\cic{J}= \cic{v}n-d\nabla n\\ 
 \cic{v} n \nabla n= \cic{v} \nabla \frac{n^2}{2}=\nabla \cic{v} \frac{n^2}{2}
 \textrm{(Hp. incompressible flow )} \dive \cic{v}=0 \end{cases} \]
  
 \[  \RA \int_V \cic{J} \cdot \nabla n \, d^3\cic{r}=\int_V \big( \cic{v} \cdot n \, \nabla n- D (\nabla n)^2 \big) \, d^3\cic{r}= 
 \int_V \big( \nabla \cic{v} \frac{n^2}{2}- D (\nabla n)^2 \big) \, d^3\cic{r}=\]
 By Gauss theorem we get: 
 \[= \oint \cancelto{x\in d\Omega \RA J=0}{\frac{ \cic{v} \, n^2}{2} \cdot d\cic{A}} -\int \underbrace{D (\nabla n)^2\,d^3\cic{r}}_{Dissipation}
 \geq \int D (\nabla n_0)^2\,d^3\cic{r}\]
 \[ \int \cic{J} \nabla\,n d^3\cic{r} \geq -\int D (\nabla n_0)^2 d^3\cic{r}= 
 \int \cic{J}_0 \nabla n_0 d^3 \cic{r}\]
 Where $\cic{J}_0$ is the flux with the same B.C. but with $\cic{v} \equiv 0$
\\ \hl{--Check------ --Insert drawing from pdfs} \\


 \fbox{\begin{minipage}{40em}
\subsubsection{Theory Remark: Dissipation minimum theorem \hl{find reference (different from Helmoltz minimum theo)} }
General idea: we consider a diffusion problem without transport (fluid velocity inducted flow of a quantity). Basing of the \textbf{minimum lagrangian action principle}, here it will follow the \textbf{minimum} of the \textbf{total dissipation}. It can be written for many other phenomena, as current distribution in conducting medium.
-Fermat-... \hl{missing stuff. add picture} 

Indeed we can compare the flux with the same B.C. but with v=0 as : \\
$ \int \cic{J} \nabla n d^3\cic{r} \geq -\int D(\nabla n_0)^2 d^3\cic{r}=\int \cic{J}_0 \nabla n d^3 \cic{r} $ \\

$ \begin{cases} d^3\cic{r}=dA\cdot dx \\ \nabla n dx=dn \end{cases} \RA \int \underbrace{\cic{J} dA}_{=\cic{J}_{tot}} \overbrace{\nabla n dx}^{=dn}= \int \cic{J}_{tot}\,dn=\cic{J}_{tot} (n_2-n_1) \geq \cic{J}_{tot, 0} (n_2-n_1) $ \end{minipage} }



\subsubsection{Effective diffusivity in shear flow } \hl{attach pic}

Now, let us estimate the effective diffusivity if there is a shear flow: we consider a flow in which the horizontal speed $v_x$ depends only on the $y$ coordinate.
We can use an approach based on the Brownian particle theory (non linear ODE). 
Then we can estimate the RMS displacement of a particle during an interval time of length $t$  . \hl{attach pic for shear flow} \\
\[ \begin{cases} <x^2>=2\,D_{eff}\, t \\ <y^2>=2\,D\,t \end{cases} \RA 
\tau \approx \frac{a^2}{D} \pvir \Delta y \approx v \tau \]

\[ \RA D_{eff} \approx \frac{\Delta y^2}{\tau} \approx \frac{v^2 \tau^2}{\tau} \approx v^2\tau \approx v^2 \frac{\mathcolorbox{red}{a^2}}{D} \]

Remembering that Peclet number is $Pe=\frac{v\,a}{D} \RA D_{eff}\approx D\cdot Pe^2$ 

\subsection{Effective diffusivity for a chess-board-like array of 2D vortices}

\begin{figure}[ht]
\centering
\includegraphics [scale=0.5] {pic2.5effectivediffusivity.jpg}
\caption{Chess-board like array of 2D vortices}
\end{figure}

We can consider the advection by a random flow, Hp. with only one characteristic scale for the random vortices. \\

\hl{Check: Diffusion as a random walk generate instability on the long term trajectory. 
But longer term, it will just reduce itself to a difference of phase (Lorentz butterfly?) and the trajectories will become the same again. Then deviate again and so on. }

\begin{figure}[ht]
\centering
\includegraphics [scale=0.7] {pic2.5effectivediffusivity2.jpg}
\caption{Chess-board like array of 2D vortices}
\end{figure}

This problem can be solved using the result of the \textbf{percolation theory} 
\footnote[31]{See Gruzinov, Isichenko, Kalda (1991) \hl{link? or insert in appendix} }

\fbox{\begin{minipage}{40em}
\subsubsection{Theory remark: Percolation and its theory}
\emph{Wiki-Def}: Percolation (from Latin percolare, "to filter" or "trickle through") refers to the movement and filtering of fluids through porous materials. It is described by \textbf{Darcy's law}. Broader applications have since been developed that cover connectivity of many systems modeled as lattices or graphs, analogous to connectivity of lattice components in the filtration problem that modulates capacity for percolation.\\
\emph{Critical probability}: The mathematical studies of the percolation phenomena led those to be a  chapter of the \textbf{statistical mechanics}. It studies the properties of a \textbf{graph} when nodes or connections between those are removed or added. This is described by a parameter $p$, that is the probability of a connection to be open or closed. Physically the probability $p$ can be interpreted as the material porosity. It exist a threshold \emph{critical value} of the probability $p_c$ between the cases in which the fluid can percolate through the material or not. 
\footnote{ Mathematically it corresponds to the existence or not of an infinite cluster of connected sites that cross all the material or not. In this transition, the parameter $p$ assume a role that is analogue to the temperature in other models of statistical mechanics, so it is possible to define critic exponents. It is also used in ecology and biochemistry.}
\end{minipage}}

\begin{figure}[ht]
\centering
\includegraphics [scale=0.5] {pic2.8percolationcritical.jpg}
\caption{$p=0.6>p_c$ then it exist a path through the material}
\end{figure}


Here we provide just a sketch about how the advection problem is connected to the percolation problem. For 2D incompressible field, we can introduce the \emph{streamfunction} and the isolines of which are the streamlines of the flow: 
\hl{ soup sentence. must be re-organized the logic-flow of the passages below}.

\[ \cic{v}=\cic{v}(x,y) \pvir \dive {\cic{v}}=0 \RA 
\begin{cases} \cic{v}=\nabla \psi \times \hat{z} \\ \cic{v}=\rot \cic{A} \end{cases} 
\textrm{   where  } \cic{A}=\psi \hat{z} \]

Here streamlines are caracterized by having $\psi (x,y)=const. $   .\\
What we want to focus the attention on are the \emph{lengths of the streamlines}, that are the iso-lines of a random landscape \hl{explain better}.\\
We are dealing with the so-called \textbf{continuous percolation problem}.
This can be \emph{mapped} to the \emph{lattice percolation problem}. \\

We proceed in the following way:  \begin{itemize}
\item We mark all the local height maxima of the landscape. These will be the vertices of the lattice.
\item We draw the steepest ascent lines (basically will represent the gradient) from each of the saddle points of the landscape to the corresponding maxima. These will be the bonds of the lattice.
\item In order to visualize this, we can imagine the landscape being flooded by water. Here a bond is declared \emph{broken} if the corresponding saddle point is below the water level, then it's not broken when above the water level. \end{itemize}
So we got exactly the percolation problem: what is the size distribution of the emerged and  connected percolation clusters? What is the length of their perimeters, so-called \emph{hulls}?  \\
Important exponents are $\nu$ of the correlation radius $x_i$ and the fractal dimension of a hull \hl{add details and explanation, from video\\ add plots and analysis of saddle points etc..}


\begin{figure}[ht]
\centering
\includegraphics [scale=0.7] {pic2.6streamlinetopology.jpg}
\caption{Streamline topology}
\end{figure}

\subsection{Homework 1}
Estimate the effective diffusivity if the 2D flow field consiste of a regular array of narrow streams and rectangular areas with non-moving water (see figure).
\hl{add figure and better explanation}\\
\hl{solution is (will be) in the appendix}\\

Some hints:
\[\delta <<a \pvir \sqrt{D\frac{a}{v}}>>\delta \pvir a\sqrt{\frac{D}{av}}=\frac{a}{\sqrt{Pe}} \RA \delta<\frac{a}{\sqrt{Pe}} \textrm{   with   }  Pe>>1  \]

\hl{Insert appendix Peclet number from last lecture }

\[ D'_{eff} \approx \frac{a^2}{\tau} \approx a\cdot v \]
Where the characteristic time constant $\tau\approx \frac{a}{v}$. 
Here we can define the awake time $t'$ \hl{find reference}\\

\fbox{\begin{minipage}{40em}
\subsubsection{Theory remark: Ergodicity}
\emph{Wiki-Def}: ergodicity expresses the idea that a point of a moving system, either a dynamical system or a stochastic process, will eventually visit all parts of the space that the system moves in, in a uniform and random sense. This implies that the average behavior of the system can be deduced from the trajectory of a "typical" point. Equivalently, a sufficiently large collection of random samples from a process can represent the average statistical properties of the entire process.\\
This can be roughly understood to be due to a common phenomenon: the motions of particles, that are geodesics on a hyperbolic manifold, are divergent; when that manifold is compact, that is, of finite size, those orbits return to the same general area, eventually filling the entire space.\\
\emph{Necessity of Ergodic hypothesis}: in statistical mechanics we work with systems made of a number of particles on the order of Avogadro number $6.022\cdot10^{23}$. This makes impossible the exact resolution of the motion equations. The aim of the statistical mechanics is to calculate average of values, that represents the macroscopic observables (as pressure, temperature or energy).\\
Note:\\
Can be seen how in this condition happens that the time average of something is equal to its spatial average. \end{minipage}}

 \[ \frac{t'}P{t}=\frac{A_{y}}{A_{tot}} \RA t'= t \frac{A_{y}}{A_{tot}} \pvir A_y\approx \textrm{thickness}\cdot \textrm{length}=\delta \cdot a \]
 (subscript $_y$ stays for \emph{yellow}, see pic)
 \[ A_t=a^2 \RA t'= t \frac{\delta}{a} \RA 
 \avg{x^2} \approx D'_{eff} \cdot t' \approx a v t' \approx 
 a v \frac{\delta}{a}t=v\delta t = D_{eff}\cdot t \]
 Where 
 \[ D_{eff}=v \delta \approx \sqrt{D a v} = D \sqrt{P} \]
 
 See that \hl{ in shear flow is squared, in this flow is square-rooted?} 
 
 \textbf{Topology of streamlines} is an important indicator of a \emph{stationary flow}. \hl{why? idea? find and explain..}
 
 
 \end{document}
 
 
 
 
 
 
 
 
 
 
 
